\documentclass{beamer}
\mode<presentation>
{
  \usetheme{default}      % or try Darmstadt, Madrid, Warsaw, ...
  \usecolortheme{default} % or try albatross, beaver, crane, ...
  \usefonttheme{default}  % or try serif, structurebold, ...
  \setbeamertemplate{navigation symbols}{}
  \setbeamertemplate{caption}[numbered]
}

\usepackage[english]{babel}
\usepackage[utf8x]{inputenc}

\title[Synthetic Populations]{Introduction to Synthetic Populations}
\author{Oskar Allerbo}
\institute{CoeGSS / Chalmers}
\date{2015-05-31}

\begin{document}

\begin{frame}
  \titlepage
\end{frame}


\begin{frame}{Outline}
  \tableofcontents
\end{frame}

\section{Synthetic Population Generation}
\begin{frame}
  \centering
  \LARGE
  Synthetic Population Generation
\end{frame}

\subsection{Introduction}
\begin{frame}{Introduction}
\begin{itemize}
\item A synthetic population is an educated guess from aggregated
  population data of what the real population behind the data looks
  like.
\item The most common (?) method to use is
  \href{https://en.wikipedia.org/wiki/Iterative_proportional_fitting}{Iterative
    Proportional Fitting (IPF)}, but there are other techniques.
\item IPF uses 1-dimensional marginal distributions for each of the
  attributes of a population (e.g.\ age or income distributions) and a
  micro sample consisting of a subsample of typically around 5 \% of
  the population where all data is available.
\item It then scales up the micro sample to fit the marginal
  distributions, while preserving the correlations between attributes
  as good as possible.
\end{itemize}
\end{frame}


\subsection{IPF Algorithm}
\begin{frame}{IPF Algorithm}
\begin{itemize}
\item Arrange the micro sample in a multidimensional matrix with as
  many dimensions as there are attributes. The number of rows/columns
  in every dimension is the same as the number of bins for the
  corresponding attribute.
\item In the example we have two attributes, income and age, with
  three bins per attribute. (The number of bins do not have to be the
  same between attributes.)
\end{itemize}

\begin{table}
\centering
\begin{tabular}{l||c c c}
      & 0-30 & 31-60 & 61-\\
\hline \hline
Low   & & & \\
Middle& & & \\
High  & & & \\
\end{tabular}
\end{table}
\end{frame}


\begin{frame}{IPF Algorithm}
\begin{itemize}
\item For every element in the matrix write the number of individuals
  in the micro sample with the particular combination of attributes
  corresponding to the element's position.
\end{itemize}

\begin{table}
\centering
\begin{tabular}{l||c c c}
      & 0-30 & 31-60 & 61- \\
\hline \hline
Low   &  3 &  6 &  3 \\
Middle&  8 &  2 &  4 \\
High  & 10 &  3 & 11 \\
\end{tabular}
\end{table}
\end{frame}


\newcommand{\sumact}{\ensuremath{\text{Sum}_{\text{act}}}}
\newcommand{\sumreq}{\ensuremath{\text{Sum}_{\text{req}}}}
\begin{frame}{IPF Algorithm}
\begin{itemize}
\item For every row/column add the actual sum and the required sum.
  The required sums are given by the marginal distributions.
\end{itemize}

\begin{table}
\centering
\begin{tabular}{l||c c c||l|l}
      & 0-30 & 31-60 & 61- & \sumact{} & \sumreq{}\\
\hline \hline
Low      &   3 &   6 &   3 &  12 &  76 \\
Middle   &   8 &   2 &   4 &  14 &  88 \\
High     &  10 &   3 &  11 &  24 &  91 \\
\hline \hline
\sumact{}&  21 &  11 &  18 &  50 &     \\
\hline
\sumreq{}& 109 &  69 &  77 &     & 255 \\
\end{tabular}
\end{table}
\end{frame}


\begin{frame}{IPF Algorithm}
\begin{itemize}
\item For every row/column in the first dimension, multiply all
  elements in the row/column with \(\sumreq{}/\sumact{}\)
  to match actual and required sums perfectly.
\end{itemize}

\begin{table}
\centering
\begin{tabular}{l||c c c||l|l}
         &  0-30 & 31-60 & 61-   & \sumact{} & \sumreq{}\\
\hline \hline
Low      &  19.00 &  38.00 &  19.00 &  76.00 &  76 \\
Middle   &  50.29 &  12.57 &  25.14 &  88.00 &  88 \\
High     &  37.92 &  11.38 &  41.71 &  91.00 &  91 \\
\hline \hline
\sumact{}& 107.20 &  61.95 &  85.85 & 255.00 &     \\
\hline
\sumreq{}& 109    &  69    &  77    &        & 255 \\
\end{tabular}
\end{table}
\end{frame}


\begin{frame}{IPF Algorithm}
\begin{itemize}
\item For every row/column in the second dimension, multiply all
  elements in the row/column with \(\sumreq{}/\sumact{}\)
  to match actual and required sums perfectly. Then do the same thing
  for the third dimension, and the fourth, and so on.
\end{itemize}

\begin{table}
\centering
\begin{tabular}{l||c c c||l|l}
 & 0-30 & 31-60 & 61- & \sumact{} & \sumreq{}\\
\hline \hline
Low & 19.32 & 42.33 & 17.04 & 78.69 & 76 \\
Middle & 51.13 & 14.00 & 22.55 & 87.68 & 88 \\
High & 38.55 & 12.67 & 37.41 & 88.63 & 91 \\
\hline \hline
\sumact{} & 109.00 & 69.00 & 77.00 & 255.00 & \\
\hline
\sumreq{} & 109 & 69 & 77 & & 255 \\
\end{tabular}
\end{table}
\end{frame}


\begin{frame}{IPF Algorithm}
\begin{itemize}
\item Start all over again with the first dimension and repeat until
  convergence.
\end{itemize}

\begin{table}
\centering
\begin{tabular}{l||c c c||l|l}
         &   0-30 & 31-60 & 61- & \sumact{} & \sumreq{}\\
\hline \hline
Low      &  18.42 & 41.36 & 16.23 & 76.01 & 76 \\
Middle   &  51.12 & 14.35 & 22.53 & 88.00 & 88 \\
High     &  39.46 & 13.29 & 38.25 & 91.00 & 91 \\
\hline \hline
\sumact{}& 109.00 & 69.00 & 77.01 & 255.02 & \\
\hline
\sumreq{}& 109    & 69 & 77 & & 255 \\
\end{tabular}
\end{table}
\end{frame}


\begin{frame}{IPF Algorithm}
\begin{itemize}
\item After some clever rounding to integers (TODO: how?) every
  element in the matrix tells us how many synthetic people there are
  for each combination of attribute values.
\end{itemize}

\begin{table}
\centering
\begin{tabular}{l||c c c||l|l}
         &  0-30 & 31-60 & 61-   & \sumact{} & \sumreq{}\\
\hline \hline
Low      &  18   &  42   &  16   & 76 & 76 \\
Middle   &  51   &  14   &  23   & 88 & 88 \\
High     &  40   &  13   &  38   & 91 & 91 \\
\hline \hline
\sumact{}& 109   &  69   &  77   & 255 & \\
\hline
\sumreq{}& 109   &  69   &  77   & & 255 \\
\end{tabular}
\end{table}
\end{frame}


\section{Activity Assignment}
\begin{frame}
  \centering
  \LARGE
  Activity Assignment
\end{frame}


\subsection{Introduction}
\begin{frame}{Introduction}
\end{frame}

\end{document}
